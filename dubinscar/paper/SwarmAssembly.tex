%  compress using: gs -sDEVICE=pdfwrite -dCompatibilityLevel=1.4 -dNOPAUSE -dQUIET -dBATCH      -sOutputFile=foo-ShapingSwarmCompressed.pdf ShapingSwarmFrictionSharedInput.pdf
% to submit: https://ras.papercept.net/conferences/scripts/start.pl
%\documentclass[conference]{IEEEtran}
\documentclass[letterpaper, 10 pt, conference]{ieeeconf}
\IEEEoverridecommandlockouts% This command is only needed if 
                                                  % you want to use the \thanks command
%\overrideIEEEmargins                                      % Needed to meet printer requirements.
\usepackage{times}


\makeatletter 
\let\NAT@parse\undefined
\makeatother

% numbers option provides compact numerical references in the text. 
%\usepackage[numbers]{natbib}
\usepackage{multicol}
%\usepackage[bookmarks=true]{hyperref}

\usepackage{bbm}
\usepackage{calc}
\usepackage{url}
\usepackage{transparent} 
\usepackage[bookmarks=true,hidelinks]{hyperref}

\usepackage{graphicx}
\usepackage[cmex10]{amsmath}
\usepackage{bm}
\usepackage{amssymb}
\usepackage{rotating}

\usepackage{chngcntr}
\counterwithin{paragraph}{subsection} % makes paragraph depend on subsection


%\usepackage{xfrac}
\usepackage{nicefrac}
\usepackage{cite}
\usepackage[caption=false,font=footnotesize]{subfig}
\usepackage[usenames, dvipsnames]{color}
\usepackage{colortbl}
%\usepackage{caption}

%\usepackage{wrapfig}
\usepackage{overpic}
%\usepackage{subfigure}
%\usepackage{textcomp}
\graphicspath{{./pictures/pdf/},{./pictures/ps/},{./pictures/png/},{./pictures/jpg/}}
\usepackage{breqn} %for breaking equations automatically
\usepackage[ruled]{algorithm}
\usepackage{algpseudocode}
%\usepackage{algorithmic}
\usepackage{multirow}
\usepackage{todonotes}
\usepackage{authblk}
\usepackage[per-mode=symbol, detect-weight=true, binary-units=true]{siunitx}
\usepackage{rotating}
\usepackage{tikz}
\usepgflibrary{arrows}% for more options on arrows
%\newcommand{\todo}[1]{\vspace{5 mm}\par \noindent \framebox{\begin{minipage}[c]{0.98 \columnwidth} \ttfamily\flushleft \textcolor{red}{#1}\end{minipage}}\vspace{5 mm}\par}
% uncomment this to hide all red todos
%\renewcommand{\todo}{}

%% ABBREVIATIONS
\newcommand{\qstart}{q_{\text{start}}}



%% MACROS


\providecommand{\abs}[1]{\left\lvert#1\right\rvert}
\providecommand{\norm}[1]{\left\lVert#1\right\rVert}
\providecommand{\normn}[2]{\left\lVert#1\right\rVert_#2}
\providecommand{\dualnorm}[1]{\norm{#1}_\ast}
\providecommand{\dualnormn}[2]{\norm{#1}_{#2\ast}}
\providecommand{\set}[1]{\lbrace\,#1\,\rbrace}
\providecommand{\cset}[2]{\lbrace\,{#1}\nobreak\mid\nobreak{#2}\,\rbrace}
\providecommand{\lscal}{<}
\providecommand{\gscal}{>}
\providecommand{\lvect}{\prec}
\providecommand{\gvect}{\succ}
\providecommand{\leqscal}{\leq}
\providecommand{\geqscal}{\geq}
\providecommand{\leqvect}{\preceq}
\providecommand{\geqvect}{\succeq}
\providecommand{\onevect}{\mathbf{1}}
\providecommand{\zerovect}{\mathbf{0}}
\providecommand{\field}[1]{\mathbb{#1}}
\providecommand{\C}{\field{C}}
\providecommand{\R}{\field{R}}
\newcommand{\Cspace}{\mathcal{Q}}
\newcommand{\Uspace}{\mathcal{U}}
\providecommand{\Fspace}{\Cspace_\text{free}}
\providecommand{\Hcal}{$\mathcal{H}$}
\providecommand{\Vcal}{$\mathcal{V}$}
\DeclareMathOperator{\conv}{conv}
\DeclareMathOperator{\cone}{cone}
\DeclareMathOperator{\homog}{homog}
\DeclareMathOperator{\domain}{dom}
\DeclareMathOperator{\range}{range}
\DeclareMathOperator{\sign}{sgn}
\providecommand{\polar}{\triangle}
\providecommand{\ainner}{\underline{a}}
\providecommand{\aouter}{\overline{a}}
\providecommand{\binner}{\underline{b}}
\providecommand{\bouter}{\overline{b}}
\newcommand{\D}{\nobreakdash-\textsc{d}}
%\newcommand{\Fspace}{\mathcal{F}}
\providecommand{\Fspace}{\Cspace_\text{free}}
\providecommand{\free}{\text{\{}\mathsf{free}\text{\}}}
\providecommand{\iff}{\Leftrightarrow}
\providecommand{\subinner}[1]{#1_{\text{inner}}}
\providecommand{\subouter}[1]{#1_{\text{outer}}}
\providecommand{\Ppoly}{\mathcal{X}}
\providecommand{\Pproj}{\mathcal{Y}}
\providecommand{\Pinner}{\subinner{\Pproj}}
\providecommand{\Pouter}{\subouter{\Pproj}}
\DeclareMathOperator{\argmax}{arg\,max}
\providecommand{\Aineq}{B}
\providecommand{\Aeq}{A}
\providecommand{\bineq}{u}
\providecommand{\beq}{t}
\DeclareMathOperator{\area}{area}
\newcommand{\contact}[1]{\Cspace_{#1}}
\newcommand{\feasible}[1]{\Fspace_{#1}}
\newcommand{\dd}{\; \mathrm{d}}
\newcommand{\figwid}{0.22\columnwidth}
\newcommand{\TRUE}{\textbf{true}}
\newcommand{\FALSE}{\textbf{false}}
\newcommand{\xupdownarrow}[1]{%
{\left\updownarrow\vbox to #1{}\right.\kern-\nulldelimiterspace}
}
%\newcommand{\xleftrightarrow}[1]{%
%  {\left\leftrightarrow\vbox to #1{}\right.\kern-\nulldelimiterspace}
%}
\DeclareMathOperator{\atan2}{atan2}
\allowdisplaybreaks

\newtheorem{theorem}{Theorem}
\newtheorem{definition}[theorem]{Definition}
\newtheorem{lemma}[theorem]{Lemma}




% paper title
\title{\LARGE \bf 2D Swarm Assembly of Rigid Objects using Uniform Inputs}

% You will get a Paper-ID when submitting a pdf file to the conference system
\author{Sheryl Manzoor, and Aaron T. Becker% <-this % stops a space
\thanks{*This work was supported by the National Science Foundation under Grant No.\ \href{http://nsf.gov/awardsearch/showAward?AWD_ID=1553063}{ [IIS-1553063]} and \href{http://nsf.gov/awardsearch/showAward?AWD_ID=1619278}{[IIS-1619278]}.}% <-this % stops a space
\thanks{Authors are with the Department of Electrical and Computer Engineering,  University of Houston, Houston, TX 77204 USA        {\tt\small  \{smanzoor2, atbecker\}@uh.edu}}%
}

\begin{document}

\maketitle
\thispagestyle{empty}
\pagestyle{empty}
%
\begin{abstract}
Imagine hundreds of particles actuated by an external magnetic field to build a component deep inside the human body.
If these particles could be small enough to pass through tiny blood vessels, they could be delivered almost anywhere in the body.
Though, the tiny size of these particles would limit the amount of force any one particle could exert, large numbers of particles could exert larger forces.
These forces could potentially be used to construct objects.  Unfortunately, the small size of the particles limits the capabilities of individual particles.
This paper considers a subset of this problem.  The particles are actuated in 2D by a uniform global external field (e.g. gravitational or magnetic fields), such that they all move in the same direction.
This paper provides a controller for multi-part assembly that computes the optimal shortest path for each component for assembly, and steers the swarm to push the component along the path.
The algorithm is validated through simulation.
\end{abstract}

\section{Introduction}\label{sec:Intro}


\section{Related Work}\label{sec:related}
% start with related work from Shiva's paper on pushing
pushing control in 2D has a long history (B. Donald, Kevin Lynch, to Shiva)

uniform controls and why

optimal control: dubins car and extentions


\section{Model}\label{sec:model}


\section{Methods}\label{sec:methods}

calculate the shortest path

figure of the 6 candidate Dubins paths for the objects,

calculate collisions, iterate path parameters to search for path

figure of the collisions and the updated path


controller: 

figure showing the path, the object, the closest point, the mean and variance of the swarm

\section{Simulation Results}\label{sec:simulation}

0.) pushing an object along a trajectory
FIgure: trace of the push
Figure: plot of error for several starting locations of the object.

1.) peg-in-hole task (one object is fixed)
figure: traces from several starting locations
plot: success rate \& timing as a function of number of robots
plot: success rate \& timing as a function of object mass

2.) two-part assembly  (two moveable parts)
figure: traces from several starting locations
plot: success rate \& timing as a function of number of robots
plot: success rate \& timing as a function of object mass


\section{Conclusion}\label{sec:conclusion}

Extensions to multi-part assembly.
Extensions to 3D
Hardware implementations


\bibliographystyle{IEEEtran}
\bibliography{IEEEabrv,SwarmAssembly}


% Uncomment to add appendix:
%\input{appendix}

\end{document}