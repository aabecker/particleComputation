% TODO: 
%  -- redo all images using vector images and bold colors
% -- insert Dallas coauthors and their work
% -- load our code onto Matlab Central.
% -- redo image 1 with an overlay of the part to generate (with #s, and arrows). Show the factory in a state of assembly
%
\pdfobjcompresslevel=0  % I had adobe error 131, and this removed it: http://tex.stackexchange.com/questions/64448/how-to-overcome-acrobat-reader-error-131-with-a-pdflatex-doc
\documentclass[letterpaper, 10 pt, conference]{ieeeconf}
\IEEEoverridecommandlockouts
\usepackage{calc}
\usepackage{url}
\usepackage{hyperref}
\hypersetup{
  colorlinks =true,
  urlcolor = black,
  linkcolor = black
}
\usepackage{graphicx}
\usepackage[cmex10]{amsmath}
\usepackage{amssymb}
\usepackage{rotating}
\usepackage{booktabs}

\usepackage{rotating}
\usepackage{nicefrac}
\usepackage{cite}
\usepackage[caption=false,font=footnotesize]{subfig}
\usepackage[usenames, dvipsnames]{color}
\usepackage{colortbl}
\usepackage{overpic}
\graphicspath{{./pictures/pdf/},{./pictures/ps/},{./pictures/png/},{./pictures/jpg/}}
\usepackage{breqn} %for breaking equations automatically
\usepackage[ruled]{algorithm}
\usepackage{algpseudocode}
\usepackage{multirow}



\usepackage{bm}   % boldface math type
\usepackage{soul}  % for highlighting
\newcommand{\todo}[1]{\vspace{5 mm}\par \noindent \framebox{\begin{minipage}[c]{0.98 \columnwidth} \ttfamily\flushleft \textcolor{red}{#1}\end{minipage}}\vspace{5 mm}\par}
% uncomment this to hide all red todos
%\renewcommand{\todo}{}

%% ABBREVIATIONS
\newcommand{\figwid}{0.22\columnwidth}
\DeclareMathOperator{\atan2}{atan2}
\newtheorem{theorem}{Theorem}
\newtheorem{definition}[theorem]{Definition}
\newtheorem{lemma}[theorem]{Lemma}
\newtheorem{corollary}[theorem]{Corollary}







\begin{document}
%%%%%%%%%%%%% For debugging purposes, I like to display the TOC
\tableofcontents
\setcounter{tocdepth}{3}
\newpage
\mbox{}
\newpage
%%%%% END TOC %%%%%%%%%%%%%%%%%%%%%%%%%%%%%%%%%%%%%%%

\title{\LARGE \bf 
Parallel Self-Assembly under Uniform Control Inputs 
}
\author{Sheryl Manzoor and Aaron T. Becker% <-this % stops a space
\thanks{*This work was supported by the National Science Foundation under Grant No.\ \href{http://nsf.gov/awardsearch/showAward?AWD_ID=1553063}{ [IIS-1553063]} and \href{http://nsf.gov/awardsearch/showAward?AWD_ID=1619278}{[IIS-1619278]}.}% <-this % stops a space
\thanks{S.~Manzoor and A.~Becker are with the Department of Electrical and Computer Engineering,  University of Houston, Houston, TX 77204 USA        {\tt\small  \{smanzoor2, atbecker\}@uh.edu}}%
}
\maketitle


\begin{abstract} 
We present fundamental progress on parallel self-assembly using large swarms of micro-scale particles
in complex environments, controlled not by individual navigation, but by a uniform global, external force with the same effect on each particle.
Consider a 2D grid world, in which all obstacles and particles are unit squares,
and for each actuation, robots move maximally until they collide with an obstacle or another robot. 
We present algorithms that, given an  arbitrary 2D structures, design an obstacle layout.
 When actuated, this layout generates copies of the 2D structure.
We analyze the spatial and time complexity of the factory layouts. 
We present hardware results on both a macro-scale, gravity-based system and a milli-scale, magnetically actuated system.








\end{abstract}

%%% The paper
%###############################################################
\section{Introduction}\label{sec:Intro}
%###############################################################

One of the exciting new directions of robotics is the design and development
of micro- and nanorobot systems, with the goal of letting a massive swarm of robots
perform complex operations in a complicated environment. Due to scaling 
issues, individual control of the involved robots becomes physically impossible:
while energy storage capacity drops with the third power of robot size,
medium resistance decreases much slower. As a consequence,
current micro- and nanorobot systems with many robots are steered and
directed by an external force that acts as a common control signal~\cite{Donald2013,Chiang2011,Hsi-Wen2012,Diller2013,Jing2013,Ou2013,Lanauze2013}.
These common control signals include global magnetic or electric fields,
chemical gradients, and turning a light source on and off.  

 \subsection{Selective Control with Global Inputs}
Clearly, having only one global signal that uniformly affects all robots at once
poses a strong restriction on the ability of the swarm to perform complex operations.
This control symmetry can be broken using interactions between the robot swarm
and obstacles in the environment. The key challenge is to establish
if interactions with obstacles are sufficient to perform complex operations, ideally by analyzing the complexity of possible logical operations.
 In previous work \cite{Becker2013f,Becker2014,Becker2014a},
we were able to demonstrate how a subset of logical functions can be implemented;
however, devising a fan-out gate (and thus the ability to replicate and copy information)
appeared to be prohibitively challenging. In this paper, we resolve this crucial question by
showing that only using unit-sized robots is insufficient for achieving computational
universality. Remarkably, adding a limited number of domino-shaped objects {\em is sufficient}
to let a common control signal, mobile particles, and unit-sized obstacles
simulate a computer. While this does not imply that large-scale computational 
tasks should be run on these particle computers instead of current electronic
devices, it establishes that future nano-scale systems are able to perform
arbitrarily complex operations {\em as part of the physical system}, instead
of having to go through external computational devices.


%\begin{figure}
 %  \centering
%\begin{overpic}[width =\columnwidth]{7tilefactory.jpg}
%\end{overpic}
%\caption{\label{fig:7tilefactory}A seven tile factory.  Each particle is actuated simultaneously by the same global control field.  The factory (black tiles) is designed so each clockwise control input assembles another component.
%}
%\end{figure}

\begin{figure}

\centering
\subfloat[]{\includegraphics[width=3.1in]{fig1a.pdf}} 
\label{fig:fig1}
\newline
\subfloat[]{\includegraphics[width=3.1in]{fig1b1.pdf}}
\label{fig:fig2}
\caption{(a) A milli-scale magnetic based prototype.
 (b) A seven tile factory. Each particle is actuated simultaneously by the same global field. The factory is designed so each clockwise control input assembles another component.}
\label{fig:1} 

\end{figure}






%   \begin{figure}
%   \centering
%   \href{http://youtu.be/EJSv8ny31r8}{
%\begin{overpic}[width =\columnwidth]{DSC_0093lowres.JPG}%\put(30,-7){ $m=1$, partition 1}
%\end{overpic}}
%\caption{\label{fig:prototype}
%Gravity-fed hardware implementation of  particle computation.  The reconfigurable prototype is setup as a {\sc fan-out} gate using a 2$\times$1 robot (white). This paper proves that such a gate is impossible using only 1$\times$1 robots. \href{http://youtu.be/EJSv8ny31r8}{See the demonstrations in the video attachment \url{http://youtu.be/EJSv8ny31r8}.} }
%\vspace{-1em}
%\end{figure}

 \subsection{Model}
  
This paper builds on the techniques for controlling many simple robots with uniform control inputs presented in \cite{Becker2013f,Becker2014,Becker2014a}, using the following rules:
\begin{enumerate}
\item A planar  grid \emph{workspace} $W$ is filled with a number of unit-square robots (each occupying one cell of the grid)  and some fixed unit-square blocks.  Each unit square in the workspace is either  \emph{free}, which a robot may occupy or \emph{obstacle} which a robot may not occupy.  Each square in the grid can be referenced by its Cartesian coordinates $\bm{x}=(x,y)$.
\item All robots are commanded in unison: the valid commands are  ``Go Up" ($u$), ``Go Right" ($r$), ``Go Down" ($d$), or ``Go Left" ($l$).  
\item Robots all move in the commanded direction until they 
	\begin{enumerate}
		\item hit an obstacle 
		\item hit a stationary robot. 
		\item share an edge with a compatible robot
	\end{enumerate}
	If a robot shares an edge with a compatible robot the two robots bond and from then on move as a unit.
A \emph{move sequence} $\bm{m}$ consists of an ordered sequence of moves $m_k$, where each $m_k\in\{u,d,r,l\}$  A representative move sequence is $\langle u,r,d,l,d,r,u,\ldots\rangle$. We assume the area of $W$ is finite and issue each command long enough for the robots to reach their maximum extent.
\end{enumerate}

%###############################################################
\section{Related Work}\label{sec:RelatedWork}
%###############################################################
Our efforts have similarities with \emph{mechanical computers},  computers
constructed from mechanical, not electrical components. For a fascinating
nontechnical review, see \cite{McCourtney1999}.  These devices have a rich
history, from the \emph{Pascaline}, an adding machine invented in 1642 by a
nineteen-year old Blaise Pascal; Herman Hollerith's punch-card tabulator in
1890; to the mechanical devices of IBM culminating in the 1940s.  These devices
used precision gears, pulleys, or electric motors to carry out calculations.
Though our {\sc Grid-World} implementations are rather basic, 
we require none of these precision elements---merely unit-size obstacles and particles.
%Can we call these robots? Indeed, the ENIAC itself was labelled a robot by the
%associated press when it was announced in the 1940s. 

\subsection{Sliding-Block Puzzles}
Sliding-block puzzles use rectangular tiles that are constrained to move in a 2D workspace. The objective is to move one or more tiles to desired locations. They have a long history.
Hearn \cite{hearn2005complexity} and Demaine \cite{Demaine2009} showed tiles can be arranged to create logic gates, and used this technique to prove {\sc pspace} complexity for a variety of sliding-block puzzles.  Hearn expressed the idea of building computers from the sliding blocks---many of the logic gates could be connected together, and the user could propagate a signal from one gate to the next by sliding intermediate tiles.  This requires the user to know precisely which sequence of gates to enable/disable.  In contrast to such a hands-on approach, with our architecture we can build circuits, store parameters in memory, and then actuate the entire system in parallel using a global control signal.

\subsection{Other Related Work on Programmable Matter}
Clearly there is a wide range of interesting scenarios for developing approaches to programmable matter.
One such model is the \emph{abstract Tile-Assembly Model} (aTAM) by Winfree~\cite{Winf98,WLWS98,LaWiRe99}, which has 
sparked a wide range of theoretical and practical research. In this model, unit-sized pixels (``tiles'')
interact and bond with the help of differently labeled edges, eventually composing complex assemblies.
Even though the operations and final objectives in this model are quite different from our particle computation with global
inputs (e.g., key features of the aTAM are that tiles can have a wide range of different edge types, and
that they keep sticking together after bonding), there is
a remarkable geometric parallelism to a key result of our present paper:
While it is widely believed that at the most basic level of interaction (called {\em temperature 1}),
computational universality {\em cannot be achieved}~\cite{LSAT1,ManuchTemp1,IUNeedsCoop} in the aTAM with only unit-sized pixels, 
very recent work~\cite{fhp+-ucapt-15} shows that computational universality {\em can be achieved} as soon as even slightly bigger tiles are used. 
This resembles the results of our paper, which shows that unit-size particles are insufficient for universal computation, while employing bigger particles suffices


%###############################################################
%\section{Theory of Globally Controlled Polyomino Assembly}\label{sec:Theory}
\section{Theory: Polyomino Assembly by Global Control}\label{sec:Theory}
%###############################################################

This section explains how to design factories that build arbitrary-shaped 2D polyominoes.
 We first assign species to individual tiles of the polyomino, second discover a build path, and finally build an assembly line of factory components that each add one tile to a partially assembled polyomino and pass the polyomino to the next component.



\subsection{Model}\label{subsec:model}
Assume the following rules:
%\begin{enumerate}
1.) A planar  grid \emph{workspace} $W$ is filled with a number of unit-square robots (each occupying one cell of the grid)  and some fixed unit-square blocks.  Each unit square in the workspace is either  \emph{free}, which a particle may occupy or \emph{obstacle} which a robot may not occupy.  Each square in the grid can be referenced by its Cartesian coordinates $\bm{x}=(x,y)$.
2.) All particles are commanded in unison: the valid commands are  ``Go Up" ($u$), ``Go Right" ($r$), ``Go Down" ($d$), or ``Go Left" ($l$).  
3.) Particles all move until they  hit an obstacle, hit a stationary particle, or share an edge with a compatible particle.
	%\begin{enumerate}
%		\item hit an obstacle 
%		\item hit a stationary particle
%		\item share an edge with a compatible particle
%	\end{enumerate}
If a particle shares an edge with a compatible robot the two robots bond and from then on move as a unit.
A \emph{move sequence} $\bm{m}$ consists of an ordered sequence of moves $m_k$, where each $m_k\in\{u,d,r,l\}$  A representative move sequence is $\langle u,r,d,l,d,r,u,\ldots\rangle$. We assume the area of $W$ is finite and issue each command long enough for the robots to reach their maximum extent.
%\end{enumerate}


%###############################################################
\subsection{Arbitrary 2D shapes require two particle species}\label{subsec:RobotSpecies}
%###############################################################
A \emph{polyomino} is a 2D geometric figure formed by joining one or more equal squares edge to edge. Polyominoes have \emph{four-point connectivity}: a 4-connected square is a neighbor to every square that shares an edge with it.


\begin{lemma}
  Any polyomino can be constructed using just two species
  \end{lemma}
\begin{proof} 
Label a grid with an alternating pattern like a checkerboard.  Any desired polyomino can be constructed on this checkerboard, and all joints are between dissimilar species.
  An example shape is shown in Fig.~\ref{fig:Grid}. Red and blue colors are used to indicate particles of different species.
  \end{proof}

   \begin{figure}
   \centering
   \vspace{0.2em}
\begin{overpic}[width =.8\columnwidth]{Grid2.pdf}
\end{overpic}
\caption{\label{fig:Grid}Any polyomino can be constructed with two compatible robot species.  
}
\end{figure}

  
  The sufficiency of two species to construct any shape gives many options for implementation.  The two species could correspond to any gendered connection, 
including ionic charge, magnetic polarity, or hook-and-loop type fasteners. Large populations of these two species can then be stored, like two-part epoxy, in separate hoppers, and only assemble when dissimilar particles come in contact.




%###############################################################
\subsection{Complexity Handled in This Paper}\label{sec:ComplexityHandled}
%###############################################################

Different 2D part geometries are more difficult to construct than others.  Fig.~\ref{fig:IncreasingDifficulty} shows parts with increasing  complexity. 

   \begin{figure}
   \centering
\begin{overpic}[width =\columnwidth]{IncreasingDifficulty3.pdf}
\end{overpic}\vspace{-2em}
\caption{\label{fig:IncreasingDifficulty}Polyomino parts. Assembly difficulty increases from left to right.
}
\end{figure} 
Label the first particle in the assembly process the \emph{seed particle}. 
 Part 1 is shaped as a `\#' symbol.  Though it has an interior hole, any of the 16 particles could serve as the seed particle, and the shape could be constructed around it.  The second shape is a spiral, and must be constructed from the inside-out.  If the outer spiral was completed first, there would be no path to add particles to finish the interior because added particles would have to slide past compatible particles.  Increasing the number of species would not solve this problem, because there is a narrow passage through the spiral that forces incoming parts to slide past the edges of all the bonded particles.
The third shape contains a loop, and the interior must be finished before the loop is closed.
Shape 4 is the combination of a left-handed and a right-handed spiral.
Adding one particle at a time in 2D cannot assemble this part, because each spiral must be constructed from the inside-out.  
 Instead, this part must be divided into sub-assemblies that are each constructed, and then combined.
 Shape 5 contains compound overhangs, and may be impossible to construct with additive 2D manufacturing using only two species.
 The algorithms in this paper detect if the desired shape can be constructed one particle at a time.  
 If so, a build order is provided, and a factory layout is designed.


% A polyomino is said to be \emph{column convex} if each column has no holes. Similarly, a polyomino is said to be row convex if each row has no holes. A polyomino is said to be \emph{convex} if it is row and column convex.
%
%\begin{lemma}\label{lemma:convexonjectsCanbeConstructedAdditively}
%Any convex polyomino can be constructed by adding one particle at a time
%\end{lemma}
%\begin{proof}
%Select any pixel as the \emph{seed block}, or root node.  Perform a breadth-first search starting at the seed block, labelling each block in the order they are expanded.  Constructing the shape according to the ordering ensures that the polyomino is convex at every step of construction.
%\end{proof}

%The proof of \ref{lemma:convexonjectsCanbeConstructedAdditively} assumes the existence of fixtures for assembly.
%\todo{describe fixtures for adding one particle at a time}

%Some non-convex polynominos cannot be constructed one particle at a time, as illustrated in Fig. ~\ref{fig:IncreasingDifficulty}.    For instance, a polynomino consisting of a clockwise and a counterclockwise square spiral, joined at the ends with a gap of one unit between the spirals must be constructed by first assembling each spiral, and then combining the sub assemblies.




%###############################################################
\subsection{Discovering a Build Path}
%###############################################################

Given a polyomino, Alg.~\ref{alg:FindBuildPath} determines if the polyomino can be built by adding one component at a time.
 The  problem of determining a build order is difficult because there are $O(n!)$ possible build orders, and many of them may  violate the constraints given in Section \ref{subsec:model}.  
 Each new tile must have a straight-line path to its goal position in the polyomino that does not collide with any other particle, does not slide past an opposite species of tile, and terminates in a mating configuration with an opposite species tile.
However, as in many robotics problems, the inverse problem of deconstruction is easier than the forward problem of construction.  

\begin{algorithm}
\newcommand\algotext[1]{\end{algorithmic}#1\begin{algorithmic}[1]}
%\begin{algorithmic}[1]
%\scriptsize 
\caption{\sc {FindBuildPath}($\mathbf{P})$   \label{alg:FindBuildPath}}
$\mathbf{P}$ is the $x,y$ coordinates of a 4-connected polyomino. % that has at least a 1-tile empty border.
Returns $ \mathbf{C} $, $ \mathbf{c} $ and $\mathbf{m}$ where $ \mathbf{C} $ contains sequence of polyomino coordinates, $ \mathbf{c} $ is a vector of color labels, and $\mathbf{m}$ is a vector of directions for assembly.
\begin{algorithmic}[1]

\State\hbox{$ \mathbf{c}\leftarrow${\sc{LabelColor}}($\mathbf{P}$)}
\State $\{\mathbf{C},\mathbf{m} \}= ${\sc {Decompose}}$(\mathbf{P},\mathbf{c})$
\State \Return $\{ \mathbf{C},\mathbf{c}, \mathbf{m} \} $ 
\end{algorithmic}
\end{algorithm} 

   \begin{figure}
   \centering
\begin{overpic}[width =\columnwidth]{DeconstructionOrderMattersSlide.pdf}
\end{overpic}\vspace{-2em}
\caption{\label{fig:DeconstructionOrderMatters} Deconstruction order matters if loops are present.  Loops occur when the 8-connected freespace has more than one connected component.  In the top row, the green tile is removed first, resulting in a polyomino that cannot be decomposed. However, if the bottom right tile is removed first, deconstruction is possible.
}
\end{figure} 

Alg.~\ref{alg:FindBuildPath}  first assigns each tile in the polyomino a color, then calls the recursive function {\sc {Decompose}}, which returns either a build order of polyomino coordinates and the directions to build, or an empty list if the part cannot be constructed.  
{\sc {Decompose}} starts by calling the function {\sc {Erode}}.  {\sc {Erode}} first counts the number of components in the 8-connected freespace. An 8-connected square is a neighbor to every square that shares an edge or vertex with it. If there is more than one connected component, the polyomino contains loops.  
 {\sc {Erode}} maintains an array of the remaining tiles in the polyomino $\mathbf{R}$. 
 In the inner \textit{for loop} at line  \ref{alg:line:forloopTotryremovinEachTileERODE}, a temporary array $\mathbf{T}$ is generated that contains all but the $j$th tile in $\mathbf{R}$.
This \textit{for loop} simply checks (1) if the $j$th tile can be removed along a straight-line path without  colliding with any other particle or sliding past an opposite species of tile in line \ref{alg:line:checkpathtileERODE},  (2) that its removal does not fragment the remaining polyomino into more than one piece in line \ref{alg:line:NumConnectedCompERODE}, and (3) that its removal does not break a loop in line \ref{alg:line:Num8ConnectedCompERODE}. 
If no loops are present, this algorithm requires at most  $n/2 (1 + n)$ iterations, because there are $n$ particles to remove, and each iteration considers one less particle than the previous iteration.

Polyominoes with loops require care, because decomposing them in the wrong order can make disassembly impossible, as shown in Fig.~\ref{fig:DeconstructionOrderMatters}.
If loops exist then  {\sc {Erode}} may return only a partial decomposition, so {\sc {Decompose}} must then try every possible break point and recursively call {\sc {Decompose}} until either a solution is found, or all possible decomposition orders have been tested.  The worst-case number of function calls of  {\sc {Decompose}}  are proportional to the factorial of the number of loops, $O( |\text{\sc 8-ConnComp}(\neg\mathbf{P})| !)$. Though large, this is much less than $O(n!)$.

\begin{algorithm}
\newcommand\algotext[1]{\end{algorithmic}#1\begin{algorithmic}[1]}
%\scriptsize 
\caption{\sc {Erode}($\mathbf{P},\mathbf{c})$   \label{alg:Erode}}
$\mathbf{P}$ is the $x,y$ coordinates of a 4-connected polyomino  and $ \mathbf{c} $ is a vector of color labels.
Returns $ \mathbf{R} $, $ \mathbf{C} $, $\mathbf{m}$, and $\mathbf{\ell}$ where $ \mathbf{R} $  is a list of coordinates of the remaining polyomino, $ \mathbf{C} $ contains sequence of tile coordinates that were removed,   $\mathbf{m}$ is a vector of directions for assembly, and $\mathbf{\ell}$ if loops were encountered. $\mathbf{d} \gets\{u,d,l,r\}$
\begin{algorithmic}[1]

\State\hbox{$\mathbf{C} \leftarrow \{\}, \mathbf{m} \leftarrow \{\}, \mathbf{\ell} \gets \textrm{\sc False}, \mathbf{R} \gets \mathbf{P}$}
\State $w \gets |\text{\sc 8-ConnComp}(\neg\mathbf{R})|$ 
\For{$i\leftarrow 1, i <  |\mathbf{P}| $}
\State  \emph{successRemove} $\gets$ {\sc False}
\For{$j\leftarrow 1, j \le  |\mathbf{R}| $}
\State $\mathbf{p} \gets \mathbf{R}_j,  \mathbf{T} \gets  \mathbf{R}  \backslash   \mathbf{R}_j$

\For{$ k \leftarrow 1, k \le  4$   \label{alg:line:forloopTotryremovinEachTileERODE} }
\If{{\sc CheckPathTile}($\mathbf{T},\mathbf{p}, \mathbf{d}_k, \mathbf{c}$) \label{alg:line:checkpathtileERODE} \textbf{and}
\\ \textbf{~~~~~~~~~~~~~~}
$1 = |\text{\sc 4-ConnComp}(\mathbf{T})|$  \label{alg:line:NumConnectedCompERODE}}
\If{$w = |\text{\sc 8-ConnComp}(\neg\mathbf{T})|$  \label{alg:line:Num8ConnectedCompERODE}}

\State  $\mathbf{R} \gets \mathbf{T}$,   \emph{successRemove} $\gets$ {\sc True}
\State  $\mathbf{C}_{ 1+|\mathbf{R}|} \gets \mathbf{p},  \mathbf{m}_{ |\mathbf{R}|}  \gets \mathbf{d}_k$
\Else { $  \mathbf{\ell} \gets \textrm{\sc True}$}
\EndIf
\State \textbf{break}
\EndIf
\EndFor
\EndFor
\If {  \emph{successRemove} $=$ {\sc False}}
\State  \hbox{$\mathbf{C} \leftarrow \{\}, \mathbf{m} \leftarrow \{\}$}
\State \textbf{break}
\EndIf
\EndFor
\If {$ |\mathbf{R}| = 1$}
\State  $\mathbf{C}_{ 1} \gets \mathbf{R}_1 $
\EndIf
\State \Return $\{ \mathbf{R},\mathbf{C}, \mathbf{m}, \ell \}$ 
\end{algorithmic}
\end{algorithm} 









\begin{algorithm}
\newcommand\algotext[1]{\end{algorithmic}#1\begin{algorithmic}[1]}
%\scriptsize 
\caption{\sc {Decompose}($\mathbf{P},\mathbf{c})$   \label{alg:Decompose}}
$\mathbf{P}$ is the $x,y$ coordinates of a 4-connected polyomino and $ \mathbf{c} $ is a vector of color labels.
Returns $ \mathbf{C} $ and $\mathbf{m}$ where $ \mathbf{C} $ contains sequence of polyomino coordinates and $\mathbf{m}$ is a vector of directions for assembly. $\mathbf{d} \gets\{u,d,l,r\}$
\begin{algorithmic}[1]

\State $ \{ \mathbf{R},\mathbf{C}, \mathbf{m}, \ell \} \gets ${\sc {Erode}}$(\mathbf{P},\mathbf{c})$
\If {$|  \mathbf{R} | = 0 \textbf{ or } \neg \ell$}
\State \Return $\{ \mathbf{C},\mathbf{m} \}$ 
\EndIf

\For{$j\leftarrow 1, j \le  |\mathbf{R}| $}
\State $\mathbf{p} \gets \mathbf{R}_j,  \mathbf{T} \gets  \mathbf{R}  \backslash   \mathbf{R}_j$

\For{$ k \leftarrow 1, k \le  4$   \label{alg:line:forloopTotryremovinEachTileDecompose} }
\If{{ ( \sc CheckPathTile}($\mathbf{T},\mathbf{p}, \mathbf{d}_k, \mathbf{c}$) \label{alg:line:checkpathtileDecompose} \textbf{and }
\\ \textbf{~~~~~~~~~~~~ }
$1 = |\text{\sc 4-ConnComp}(\mathbf{T})|$)  \label{alg:line:NumConnectedCompDecompose}}
\State $\{\mathbf{C2},\mathbf{m2} \}\gets ${\sc {Decompose}}$(\mathbf{T},\mathbf{c})$
\If {$\mathbf{C2}  \ne \{\}$}
%\State  $\mathbf{C}_{ 1+|\mathbf{R}|} \gets \mathbf{p},  \mathbf{m}_{ |\mathbf{R}|}  \gets \mathbf{d}_k$
\State $\mathbf{C}_{1:|\mathbf{C2}|+1} \gets \{\mathbf{C2},\mathbf{p}\}$
\State $ \mathbf{m}_{1:|\mathbf{m2}|+1} \gets \{\mathbf{m2},\mathbf{d}_k\}$
\State \Return $\{ \mathbf{C}, \mathbf{m} \}$ 
\EndIf
\State \textbf{break}
\EndIf
%\EndIf
\EndFor
\EndFor
\State $\mathbf{C} \gets \{\}, \mathbf{m} \gets \{\}$
\State \Return $\{ \mathbf{C}, \mathbf{m} \}$ 
\end{algorithmic}
\end{algorithm} 


  
%###############################################################
\subsection{Assembling Tiles}
%###############################################################


%###############################################################
\subsubsection{Hopper Construction}\label{subsec:HopperConstruction}
%###############################################################
Two-part adhesives react when components mix.  Placing components in separate containers prevents mixing.  Similarly, storing many particles of a single species in separate containers allows controlled mixing.
%WIKI: harden by mixing two or more components which chemically react.

We can design \emph{part hoppers}, containers that store similarly labelled particles.  These particles will not bond with each other.  The hopper shown in Fig.~\ref{fig:HopperCW} releases one particle every cycle. Delay blocks are used to ensure the $n$th part hopper does not start releasing particles until cycle $n$. For ease of exposition, this paper has a unique hopper for each tile position. This enables precise positioning of different materials, but a particle logic system could use just two hoppers, similar to our particle logic systems in [9].

   \begin{figure}
   \centering
\begin{overpic}[width =\columnwidth]{hopperV4.pdf}
\end{overpic}\\ \vspace{-1em}
\caption{\label{fig:HopperCW}Hopper with five delays. The hopper is filled with similarly-labelled robots that will not combine.  Every clockwise command sequence $\langle l,u,r,d \rangle$ releases one robot from the hopper.  %\textcolor{red}{replace with new hopper design}
}
\end{figure}


\begin{figure}
   \centering
\begin{overpic}[width =\columnwidth]{24tilefactory.pdf}
\end{overpic}
\begin{overpic}[width =\columnwidth]{Spiraltilefactory.pdf}
\end{overpic}\\ \vspace{-1em}
\caption{\label{fig:24Tilefactory}A twenty-four tile factory, step 82 for a `\#' shape and a twenty-one tile factory, step 66 for a spiral (zoom in for details in this vector graphic).
}
\end{figure}







%###############################################################
\subsection{Part Assembly Jigs}\label{subsec:PartAssemblyJigs}
%###############################################################

Assembly is an iterative procedure.  
A factory layout is generated by  {\sc{BuildFactory}}($\mathbf{P}, n_c$), described in Alg.~\ref{alg:BuildFactory}. This function takes a 2D polyomino $\mathbf{P}$ and, if $\mathbf{P}$ has a valid build path, designs an obstacle layout to generate $n_c$ copies of the polyomino. A polyomino is composed of $|\mathbf{P}| = n$ tiles.  

For each tile, the function 
 {\sc{FactoryAddTile}} $(n_c,\mathbf{b}, m,C, c,w)$
  described in  Alg.~\ref{alg:FactoryAddTile}
is called to generate an obstacle configuration $\mathbf{A}$.
$\mathbf{A}$  forms a hopper that releases a particle each iteration and a chamber that temporarily holds the partially-assembled polyomino $\mathbf{b}$ and guides the new particle $C$ to the correct mating position. A 24-tile factory is shown in  Fig.~\ref{fig:24Tilefactory}.


%\todo{Sheryl, add the algorithmic environment for Build Factory}
\begin{algorithm} 
\newcommand\algotext[1]{\end{algorithmic}#1\begin{algorithmic}[1]}
%\scriptsize
\caption{ \sc{BuildFactory}($\mathbf{P}, n_c$)\label{alg:BuildFactory}}
$\mathbf{P}$ is the $x,y$ coordinates of a 4-connected polyomino.  $n_c$ is the number of parts desired. 
Returns a two dimensional array $ \mathbf{F} $ containing the factory obstacles and filled hoppers.
\begin{algorithmic}[1]
\State$\mathbf{F} \leftarrow \{\}$ \Comment{the factory obstacle array} 

\State \{$\mathbf{C},\mathbf{c}, \mathbf{m}$\} $  \leftarrow$ {\sc{FindBuildPath}}($\mathbf{P}$)
 \If{$ \{\} = \mathbf{m}$}
 \State \Return  $ \mathbf{F} $
 \EndIf 
 \State$\{ \mathbf{A}, \mathbf{b} \}\leftarrow${\sc{FactoryFirstTile}}$(n_c, \mathbf{c}_i,w)$
 \For{$i\leftarrow 2, i \le  |\mathbf{c}| )$}
 \State$\{\mathbf{A},\mathbf{b}\}\leftarrow${\sc{FactoryAddTile}}$(n_c,\mathbf{b}, \mathbf{m}_{i-1},\mathbf{C}_i, \mathbf{c}_i,w)$
 \State$ \mathbf{F} \leftarrow${\sc{ConcatFactories}}$(\mathbf{F},\mathbf{A})$
\EndFor
\State \Return  $ \mathbf{F} $
%\State{\sc{DisplayFactory}}($factoryLayout$)
\end{algorithmic}
\end{algorithm} 
 
 
 

 
 
\begin{algorithm} 
\newcommand\algotext[1]{\end{algorithmic}#1\begin{algorithmic}[1]}
%\scriptsize
\caption{\sc {FactoryAddTile}$(n_c,\mathbf{b}, m,C, c,w)$ \label{alg:FactoryAddTile}}
\begin{algorithmic}[1]
\State$
\{ \mathbf{hopper}\}\leftarrow${\sc{Hopper}}$(c,n_c,w)$
\If{ $m = d \textbf{ and } \left(     C_x  \le \max \mathbf{b}_x   
                         \textbf{ or }  C_y     < \min \mathbf{b}_y \right)  }$
    
\State$\{\mathbf{A},\mathbf{b}\}\leftarrow${\sc{downdir}}$(\mathbf{hopper},\mathbf{b},\mathbf{C})$

\ElsIf{ $m = l \textbf{ and} \left(     C_y  \le \max \mathbf{b}_y   
                         \textbf{ or }  C_x     > \max \mathbf{b}_x \right)  }$
    
\State$\{\mathbf{A},\mathbf{b}\}\leftarrow${\sc{leftdir}}$(\mathbf{hopper},\mathbf{b},\mathbf{C})$
\ElsIf{ $m = l \textbf{ and} \left(     C_x  \ge \max \mathbf{b}_x   
                         \textbf{ or }  C_y     > \max \mathbf{b}_y \right)  }$
    
\State$\{\mathbf{A},\mathbf{b}\}\leftarrow${\sc{updir}}$(\mathbf{hopper},\mathbf{b},\mathbf{C})$
\ElsIf{ $m = r \textbf{ and } \left(     C_y     \ge \min \mathbf{b}_y   
                       \textbf{ or }  C_x  < \min \mathbf{b}_x   \right)  }$
\State$\{\mathbf{A},\mathbf{b}\}\leftarrow${\sc{rightdir}}$(\mathbf{hopper},\mathbf{b},\mathbf{C})$



\EndIf

\State \Return $\{ \mathbf{A}, \mathbf{b} \}$ 

\end{algorithmic}
\end{algorithm}
 
 
 
 
 
 






%###############################################################
\section{Analysis}\label{sec:Analysis}
%###############################################################
This section analyzes the time and space required for a factory and gives simulation results.


%###############################################################
\subsection{Running Time}\label{sec:runningTime}
%###############################################################
Running a factory simulation has three phases, ramp up, production, and wind down.
During the $n-1$ \emph{ramp up}  cycles, the first polyomino is being constructed one tile at a time and no polyominoes are produced.
Clever design of delays in the part hoppers ensures no unconnected tiles are released.
During \emph{production} cycles, one  polyomino is finished each cycle.
Once the first part hopper empties, the $n-1$ \emph{wind down}  cycles each produce a complete polyomino as each successive hopper empties.
 This section analyzes running time, defined as the time required for each commanded move until all tiles are stopped.  
 We assume all tiles move unit distance in unit time.
 There are two results, the \emph{construction time}, the time required to assemble a single polyomino from scratch, and
 the \emph{cycle time}, the time required during production cycles to advance all partial assemblies one cycle.
 Since a polyomino contains $n$ tiles, the \emph{construction time} during production cycles is $n \cdot$ \emph{cycle time}.
 
Cycle time is the sum of the maximum distances moved in each direction.
 As shown in Fig.~\ref{fig:timeplot}, polyominoes shaped as a $n\times 1$ row require the longest time of $4n+16$.
Polyominoes shaped as a $1\times n$ column require the least time of $2n+16$.
 Construction time therefore requires $O(n^2)$ time.
 \begin{figure}
   \centering
\begin{overpic}[width =1\columnwidth]{maxcycleplot.pdf}
\end{overpic}
\caption{\label{fig:timeplot}Cycle time plotted against number of tiles $n$.  The cycle time is the sum time to move during the $r,d,l,u$ moves each cycle. Cycle time increases linearly and is upper bounded by row parts and lower bounded by column parts.  Total construction time for a particle is $n \cdot $ cycle time.  
}
\end{figure}


%###############################################################
\subsection{Space Required}\label{sec:requiredSpace}
%###############################################################
The space required by a factory is a function of the size of individual sub-assemblies.



\begin{align}
height(n)=
\begin{cases}
\left \lceil{\frac{n_c}{w}}\right \rceil+2(\left \lceil{\frac{n}{2}}\right \rceil+2+partrows) 
& m \in \{l, d\}\\
\left \lceil{\frac{n_c}{w}}\right \rceil+2(\left \lceil{\frac{n}{2}}\right \rceil+3+partrows) 
&  m \in \{r, u\} 
\end{cases}
\end{align}

Because a factory requires $O(n)$ rows and $O(n)$ columns, the total requires space is $O(n^2)$.
As shown in Fig.~\ref{fig:sizeplot}, the required size is  upper bounded by column-shaped polynominos and lower bounded by row-shaped polyominos, and is $O(n^2)$.

\begin{figure}
   \centering
\begin{overpic}[width =1\columnwidth]{facsizeplot1.pdf}
\end{overpic}
\caption{\label{fig:sizeplot}
Factory size grows quadratically with the number of tiles, and is upper bounded by column-shaped polynominos and lower bounded by row-shaped polyominos.
}
\end{figure}


%###############################################################
\subsection{Simulation Results}\label{sec:simResults}
%###############################################################

Algorithms  \ref{alg:FindBuildPath} through \ref{alg:FactoryAddTile}  were coded in {\sc Matlab} and are available at \cite{Manzoor2017gitAssemply}.  









%###############################################################
\section{Experiment}\label{sec:Experiment}
%###############################################################

%\subsection{Macro-scale, Gravity-Based Prototype}
%
%To demonstrate Algorithms 1-5 
%
%
%\begin{figure}
%   \centering
%\begin{overpic}[width =\columnwidth]{MacroScalePrototype}
%\end{overpic}
%\caption{\label{fig:24tilefactory}A large-scale demonstration of particle assembly using gravity as the external force and magnetic attraction between red and blue particles for assembly.
%}
%\end{figure}
%
%
%\subsection{Milli-scale, Magnetic-Based Prototype}
To demonstrate the algorithm, we developed a magnetic control stage and alginate micro-particles.

\paragraph{Experimental setup}


\begin{figure}
   \centering
\begin{overpic}[width =\columnwidth]{BastExp1.pdf}
\end{overpic}
\caption{\label{fig:Magneticstage}Experimental platform.  %\todo{describe the system}
}
\end{figure}

This stage generates a magnetic drag force by moving a permanent magnet. The permanent magnet is able to move $x, y$ direction as following two mail shafts. The permanent magnet has  T and the dimension is cm$^2$. The main channel is made up PDMS and it was filled with motility buffer. The alginate microrobot was fabricated using ~~~. After the alginate microrobots were located at each chamber in the channel, the experimental channel was located on the center of the stage where a magnet was positioned initially. The stage controller was manipulated by a C++ programming through an Arduino UNO. The channel was observed by a stereo microscope and the installed camera captured all sequent images (fps). The scheme of system is shown in Fig.~\ref{fig:Magneticstage}.

\paragraph{Experimental result}
Using one of construction maps, it is available to demonstrate the map using multiple alginate microrobots. The initial scene is shown in Fig.~\ref{fig:Construction}a and the first assemble was manipulated moving the magnet in a clockwise direction as indicated in Fig.~\ref{fig:Construction}b. The alginate microrobots moved in the oriented direction until coming into contact with an object. The final completion of a square polyomino is shown in the lower right corner in Fig.~\ref{fig:Construction}c. In addition,  other polyominoes were simultaneously being manufactured. 


\begin{figure}
   \centering
\begin{overpic}[width =\columnwidth]{BastExp2.pdf}
\end{overpic}
\caption{\label{fig:Construction}Fig. Construction of a microrobotic polyomino from four alginate artificial cells. (a) Initial position of alginate microrobots at all chambers, (b) First assemble by two microrobots from two chambers, (c) Final result of construction.
}
\end{figure}

%###############################################################
\section{Conclusion}\label{sec:Conclusion}
%###############################################################

This work, along with \cite{Becker2013f,Becker2014,Becker2014a}, introduces a
new model for additive assembly.  Interesting applications will aim at  microfluidics work.

Future work could extend Algorithms \ref{alg:FindBuildPath}--\ref{alg:FactoryAddTile} to three dimensions. 
Parts can be decomposed into subassemblies, which would enable more complex parts to be created and enable construction in logarithmic time.
    
%%

\bibliographystyle{IEEEtran}
\bibliography{IEEEabrv,../../RoboticSwarmControlLab/bib/aaronrefs}
\end{document}

