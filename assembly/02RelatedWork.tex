<<<<<<< HEAD
%###############################################################
\section{Related Work}\label{sec:RelatedWork}
%###############################################################
Our efforts have similarities with \emph{mechanical computers},  computers
constructed from mechanical, not electrical components. For a fascinating
nontechnical review, see \cite{McCourtney1999}.  These devices have a rich
history, from the \emph{Pascaline}, an adding machine invented in 1642 by a
nineteen-year old Blaise Pascal; Herman Hollerith's punch-card tabulator in
1890; to the mechanical devices of IBM culminating in the 1940s.  These devices
used precision gears, pulleys, or electric motors to carry out calculations.
Though our {\sc Grid-World} implementations are rather basic, 
we require none of these precision elements---merely unit-size obstacles and particles.
%Can we call these robots? Indeed, the ENIAC itself was labelled a robot by the
%associated press when it was announced in the 1940s. 

\subsection{Sliding-Block Puzzles}
Sliding-block puzzles use rectangular tiles that are constrained to move in a 2D workspace. The objective is to move one or more tiles to desired locations. They have a long history.
Hearn \cite{hearn2005complexity} and Demaine \cite{Demaine2009} showed tiles can be arranged to create logic gates, and used this technique to prove {\sc pspace} complexity for a variety of sliding-block puzzles.  Hearn expressed the idea of building computers from the sliding blocks---many of the logic gates could be connected together, and the user could propagate a signal from one gate to the next by sliding intermediate tiles.  This requires the user to know precisely which sequence of gates to enable/disable.  In contrast to such a hands-on approach, with our architecture we can build circuits, store parameters in memory, and then actuate the entire system in parallel using a global control signal.

\subsection{Other Related Work on Programmable Matter}
Clearly there is a wide range of interesting scenarios for developing approaches to programmable matter.
One such model is the \emph{abstract Tile-Assembly Model} (aTAM) by Winfree~\cite{Winf98,WLWS98,LaWiRe99}, which has 
sparked a wide range of theoretical and practical research. In this model, unit-sized pixels (``tiles'')
interact and bond with the help of differently labeled edges, eventually composing complex assemblies.
Even though the operations and final objectives in this model are quite different from our particle computation with global
inputs (e.g., key features of the aTAM are that tiles can have a wide range of different edge types, and
that they keep sticking together after bonding), there is
a remarkable geometric parallelism to a key result of our present paper:
While it is widely believed that at the most basic level of interaction (called {\em temperature 1}),
computational universality {\em cannot be achieved}~\cite{LSAT1,ManuchTemp1,IUNeedsCoop} in the aTAM with only unit-sized pixels, 
very recent work~\cite{fhp+-ucapt-15} shows that computational universality {\em can be achieved} as soon as even slightly bigger tiles are used. 
This resembles the results of our paper, which shows that unit-size particles are insufficient for universal computation, while employing bigger particles suffices

=======
%###############################################################
\section{Related Work}\label{sec:RelatedWork}
%###############################################################


\subsection{Microscale Biomanufacturing}
Naturally derived biomaterials as building blocks for functional materials and devices are increasingly desired because they are environmentally and biologically safer than purely synthetic materials. 
One such class of materials, polysaccharide based hydrogels, are intriguing because they can reversibly encapsulate a variety of smaller components. Many groups have termed these loaded-alginate particles as artificial cells, in that they mimic the basic structure of living cells (membrane, cytoplasm, organelles, etc.) \cite{chang2005therapeutic} [1-3]. 
Construction with these micron-sized gels has numerous applications in industry, including cell manipulation, tissue engineering, and micro-particle assembly [4-8], but requires fundamental research in biology, medicine, and colloidal science. 
While there are several methods to efficiently fabricate these particulate systems, it is still challenging to construct larger composite materials out of these units [9]. Traditional methods of assembling larger macroscale systems are unemployable due to the change of dominant forces at small length scales. 
In particular, forces due to electromagnetic interactions dominate gravitational forces at the microscale resulting in strong adhesion and sudden shifts in the position of microparts under atmospheric conditions. 
%Furthermore, analogs of basic macroscale robotic elements have not been suitably designed and commercialized. 
To form constructs out of microgels, groups have traditionally turned to non-robotic microfluidic systems that utilize a variety of actuation methods, including mechanical, optical, dielectrophoretic, acoustophoretic, and thermophoretic [10-14]. 
While each of these methods has proven to be capable of manipulating biological cells, each method has significant drawbacks that limit their widespread application. 
For example, microscale mechanical, acoustophoretic, and thermophoretic manipulation methods use stimuli that can be potentially lethal to live cells [15]. 
Furthermore, most, if not all, of these techniques require expensive equipment and lack control schemes necessary to precisely manipulate large numbers of cells autonomously.

\subsection{Control Microrobotic Swarms Using Only Global Signals}
Today one of the most exciting new frontiers in robotics is the development of micro- and nanorobotic systems, which hold the potential to revolutionize the fields of manufacturing and medicine. 
Chemist, biologist, and roboticist have shown the ability to produce very large populations (10$^3$--10$^{14}$) of small scale (10$^{-9}$--10$^{-6}$ m) robots using a diverse array of materials and techniques [16-18]. 
Untethered swarms of these tiny robots may be ideal for on-site construction of high-resolution macroscale materials and devices. 
While these new types of large-population, small-sized, robotic systems have many advantages over their larger-scale counterparts, they also present a set of unique challenges in terms of their control. 
Due to  current limitations in fabrication, micro- and nanorobots have little-to-no onboard computation, along with limited computation and communication ability [18-20]. 
These limitations make controlling swarms of these robots individually impractical. 
Thus, these robotic systems are often controlled by a uniform global external signal (e.g. chemical gradients, electric and magnetic fields), which makes motion planning for large robotic populations in tortuous environments difficult. 
We recently demonstrated that obstacles present in the workspace can break the symmetry of approximately identical robotic swarms, enabling positional configuration of robots [21]. 
 Given a large-enough free space, a single obstacle is sufficient for positional control over $N$ particles.  
This method can be used to form complex assemblies out of large swarms of mobile microrobotic building blocks, using only a single global input signal.

\subsection{Microrobot Based Microassembly}
The ability to create microrobots, and control algorithms capable of autonomous manipulation and assembly of small scale components into functional materials is currently a major manufacturing challenge [1]. 
To address this challenge, teams of microrobotic systems must work together intelligently to coordinate manipulation tasks in novel environments. 
While several microrobots capable of performing simple manipulation and assembly tasks have been reported [2-7], few have shown the ability to pattern intricate designs or assemble complex multi-component parts. 
Recently, some groups have begun to develop cell-safe magnetically actuated microrobotic systems for cell patterning, yet their method is limited in that these systems are manually controlled, not automated, and suffer from low spatial resolution [22, 23]. 
In this paper, we seek to combine the use of microscale hybrid organic/inorganic actuators along with novel swarm control algorithms for mask free programmable patterning and micro-assembly. 
Specifically, this paper applies swarm control and particle logic computations to magnetically actuate artificial cells, so as to use them as micro-scale robotic swarms, to create complex, high resolution, 2D and 3D patterns and assemblies.
>>>>>>> origin/master
